%%=============================================================================
%% Conclusie
%%=============================================================================

\chapter{Conclusie}%
\label{ch:conclusie}

% TODO: Trek een duidelijke conclusie, in de vorm van een antwoord op de
% onderzoeksvra(a)g(en). Wat was jouw bijdrage aan het onderzoeksdomein en
% hoe biedt dit meerwaarde aan het vakgebied/doelgroep? 
% Reflecteer kritisch over het resultaat. In Engelse teksten wordt deze sectie
% ``Discussion'' genoemd. Had je deze uitkomst verwacht? Zijn er zaken die nog
% niet duidelijk zijn?
% Heeft het onderzoek geleid tot nieuwe vragen die uitnodigen tot verder 
%onderzoek?

Het centrale onderzoeksvraagstuk in deze bachelorproef is: 
Hoe kan \gls{iot}-technologie worden ingezet om de bezettingsgraad van wachtruimtes in real-time te meten en voorspellende analyses toe te passen, met als doel het inzicht in bezettingsgraden te verbeteren en zo een basis te bieden voor efficiënter bezoekersbeheer? \\

De \gls{poc} toont aan dat \gls{iot}-technologieën in staat zijn om de bezetting in een wachtruimte te detecteren en real-time te registreren. Dit toont de technische haalbaarheid en de potentiële meerwaarde van een sensorgebaseerd systeem.
Bij de samenwerking met \gls{wgc} De Vlier werd het duidelijk dat bezettingsdata gebruikt kan worden als aanvulling voor bestaande data en als ondersteuning kan bieden bij zorgplanning en capaciteitsbeheer. \\

Toch is dit niet genoeg om dit systeem te beschouwen als alternatief door het gebrek aan robuustheid. Storingen en instabiliteit zorgen voor een beperkte dataset waardoor alleen een vergelijking met Vertex AI mogelijk was. Daardoor kon de validatie van de nauwkeurigheid niet worden uitgevoerd. De resultaten tonen dus aan dat de \gls{poc} haalbaar is maar dat de betrouwbaarheid een aandachtspunt blijft. \\

De bijdrage van dit onderzoek ligt in het aantonen dat een \gls{iot}-gebaseerde systeem een privacyvriendelijk en kostenefficiënt alternatief biedt voor vision-gebaseerde oplossingen. Dit kan een meerwaarde bieden voor organisaties (klein tot middengroot) die op zoek zijn naar kostenefficiënte methoden om inzicht te krijgen in wachtruimtebezetting. \\

Voor verdere onderzoek is het essentieel om langere metingen uit te voeren, sensorplaatsing te optimaliseren en stabiliteitsproblemen te reduceren. Dankzij deze aanbevelingen kunnen voorspellende modellen uitgevoerd worden zodat bezettingsanalyse kunnen bijdragen aan efficiënter bezoekers- en personeelsbeheer \\

Samenvattend kan gesteld worden dat het onderzoek gedeeltelijk succesvol was: \gls{iot}-technologie is een veelbelovende basis gebleken voor real-time monitoring, maar verdere stappen zijn nodig om de betrouwbaarheid en praktische inzetbaarheid te garanderen.
%Tegelijkertijd is de robuustheid van het systeem nog onvoldoende om van een volwaardig alternatief te spreken. Storingen in de sensoren en instabiliteit in de dataverzameling leidden tot beperkte datasets, waardoor slechts een kwalitatieve vergelijking met Google Vertex AI Occupancy Analytics mogelijk was. Een sluitende kwantitatieve validatie van de nauwkeurigheid kon daardoor niet worden uitgevoerd. De resultaten moeten dus kritisch worden geïnterpreteerd: de haalbaarheid is aangetoond, maar de betrouwbaarheid blijft een aandachtspunt. \\

%De bijdrage van dit onderzoek ligt in het aantonen dat een \gls{iot}-gebaseerde aanpak een privacyvriendelijk en kostenefficiënt alternatief kan zijn voor vision-gebaseerde oplossingen. Dit biedt meerwaarde voor organisaties die op zoek zijn naar laagdrempelige methoden om inzicht te krijgen in hun wachtruimtebezetting. \\

%Voor toekomstig onderzoek is het essentieel om langere meetcampagnes op te zetten, sensoropstellingen te verfijnen en stabiliteitsproblemen structureel te reduceren. Bovendien is er nood aan het integreren van voorspellende modellen op basis van grotere en consistente datasets, zodat wachttijdanalyses daadwerkelijk kunnen bijdragen aan efficiënter bezoekers- en personeelsbeheer. \\


\printglossaries






