%%=============================================================================
%% Voorwoord
%%=============================================================================

\chapter*{\IfLanguageName{dutch}{Woord vooraf}{Preface}}%
\label{ch:voorwoord}
\TODO
%% TODO:
%% Het voorwoord is het enige deel van de bachelorproef waar je vanuit je
%% eigen standpunt (``ik-vorm'') mag schrijven. Je kan hier bv. motiveren
%% waarom jij het onderwerp wil bespreken.
%% Vergeet ook niet te bedanken wie je geholpen/gesteund/... heeft

Het schrijven van dit voorwoord markeert het einde van mijn opleiding Toegepaste Informatica aan HOGENT. Deze opleiding was voor mij een intensief traject, waarin ik soms aan mezelf twijfelde en me afvroeg of ik het wel zou aankunnen. Dankzij doorzettingsvermogen en hard werk is het echter een bijzonder leerrijke ervaring geworden, waarin ik mijn kennis en vaardigheden kon verdiepen en succesvol toepassen binnen dit onderzoek. \\

Allereerst wil ik mijn promotor, de heer Simon De Gheselle, bedanken voor zijn begeleiding, ondersteuning en waardevolle feedback tijdens dit traject. Zijn hulp was voor mij essentieel en ik kan hem daar niet genoeg voor bedanken. Verder gaat mijn dank uit naar mijn co-promotor, de heer Lionel Anciaux, voor zijn advies en inhoudelijke steun. \\

Daarnaast wil ik WGC De Vlier en de werknemers aan de balie bedanken voor hun medewerking en ondersteuning tijdens mijn onderzoek. Mijn bijzondere dank gaat uit naar Matthias De Vos, die mijn proof-of-concept heeft geaccepteerd en mij met vertrouwen en hulp heeft ondersteund bij het opzetten ervan, evenals naar Massimo Moro voor zijn bereidwillige hulp en waardevolle inbreng. \\

Ook wil ik de lectoren van de opleiding Toegepaste Informatica bedanken voor de kennis en begeleiding die zij mij gedurende mijn studietraject hebben meegegeven. \\

Tot slot gaat mijn dank uit naar mijn familie, en in het bijzonder mijn broer Kerim Mahfoudhi, voor zijn hulp bij het opzetten van de proof-of-concept en voor zijn voortdurende steun en motivatie. \\

Omdat ik geen moedertaalspreker van het Nederlands ben, heb ik gebruik gemaakt van generatieve AI als hulpmiddel bij de taal en de structuur van deze tekst. Ook enkele codefragmenten werden met behulp van AI gegenereerd, maar telkens door mijzelf getest en aangepast. De inhoud en analyses zijn volledig mijn eigen werk. \\

Brahim Mahfoudhi

%Het implementeren van Internet of Things (IoT) om de wachttijden voor patiënten in de afdeling Accident and Emergency (A\&E) van de National Health Service (NHS) te verkorten”. Dit onderzoek werd geschreven met als doel om te bepalen of IoT een daadwerkelijke oplossing kan bieden tegen de langdurige wachttijden. Dit onderwerp heb ik gekozen samen met mijn mede-student Mohamed-Ali Kasraoui. Allereerst wil ik hem bedanken voor zijn waardevolle bijdrage, zonder welke dit onderzoek niet had kunnen plaatsvinden.

%De keuze voor het onderwerp werd sterk beïnvloed door de langdurige ziekte die ik aan het begin van mijn studies doormaakte. Het zijn voor mij zeer moeilijke jaren geweest, voor die reden was het van uiterst belang dat dit onderzoek zich op de medische sector richt. Het was voor mij van groot belang dat het onderzoek dat ik uitvoerde een waardevolle bijdrage zou leveren aan de medische wereld, waarvoor ik zeer dankbaar ben. %%TODO: ERVARING TIJDENS BACHELORPROEF

%TODO: Bedanking Lectoren, Promotor, co-promotor
%Ik wil iedere lector bedanken in deze drie geweldige jaren, ik heb dankzij voldoende 


%Ik wens u veel leesplezier toe.

%Brahim Mahfoudhi




