%%=============================================================================
%% Samenvatting
%%=============================================================================

% TODO: De "abstract" of samenvatting is een kernachtige (~ 1 blz. voor een
% thesis) synthese van het document.
%
% Een goede abstract biedt een kernachtig antwoord op volgende vragen:
%
% 1. Waarover gaat de bachelorproef?
% 2. Waarom heb je er over geschreven?
% 3. Hoe heb je het onderzoek uitgevoerd?
% 4. Wat waren de resultaten? Wat blijkt uit je onderzoek?
% 5. Wat betekenen je resultaten? Wat is de relevantie voor het werkveld?
%
% Daarom bestaat een abstract uit volgende componenten:
%
% - inleiding + kaderen thema
% - probleemstelling
% - (centrale) onderzoeksvraag
% - onderzoeksdoelstelling
% - methodologie
% - resultaten (beperk tot de belangrijkste, relevant voor de onderzoeksvraag)
% - conclusies, aanbevelingen, beperkingen
%
% LET OP! Een samenvatting is GEEN voorwoord!

%%---------- Nederlandse samenvatting -----------------------------------------
%
% TODO: Als je je bachelorproef in het Engels schrijft, moet je eerst een
% Nederlandse samenvatting invoegen. Haal daarvoor onderstaande code uit
% commentaar.
% Wie zijn bachelorproef in het Nederlands schrijft, kan dit negeren, de inhoud
% wordt niet in het document ingevoegd.

\IfLanguageName{english}{%
\selectlanguage{dutch}
\chapter*{Samenvatting}
\lipsum[1-4]
\selectlanguage{english}
}{}

%%---------- Samenvatting -----------------------------------------------------
% De samenvatting in de hoofdtaal van het document

\chapter*{\IfLanguageName{dutch}{Samenvatting}{Abstract}}

De wachttijden voor Accident and Emergency (A&E) behandelingen binnen de National Health Service (NHS)
in Engeland hebben over de vele jaren een stijgende trend vertoond. Ondanks de systematische steun van de
Britse overheid heeft de National Health Service (NHS) nog geen oplossing kunnen vinden voor dit probleem.
Echter maken patiënten zich ernstige zorgen over de wachttijden, worden onopzettelijk genegeerd en verergeren
bestaande medische aandoeningen tijdens het wachten, wat kan leiden tot ernstige complicaties. Om die
reden richt dit onderzoek zich op het beantwoorden van de volgende vraag: Kan Internet of Things (IoT) geïmplementeerd
worden om de wachttijden te verminderen en de efficiëntie van de zorgverlening te verbeteren?
Het onderzoek start met een oorzaakanalyse uitgevoerd om inzicht te krijgen in de oorzaak van de lange wachttijden.
Hierna volgt er een casestudy voor het identificeren van concrete voorbeelden in de gezondheidszorg van
IoT-implementaties om wachttijden te verminderen, de casestudy zal verder gebruikt worden voor het identificeren
van IoT-devices die gebruikt kunnen worden in de proof-of-concept. Vervolgens wordt een vergelijkende
studie uitgevoerd waarin IoT-apparaten worden geanalyseerd om het meest geschikte apparaat te identificeren.
Verder zal er een ziekenhuis geselecteerd worden voor het uitvoeren van de Proof-of-Concept. Zodra een ziekenhuis
is geselecteerd, zal er een meeting plaatsvinden om de huidige wachttijden in kaart te brengen. Hierna
zullen de verzamelde gegevens gebruikt worden om een proof-of-concept uit te werken die aantoont dat Internet
of Things (IoT) gebruikt kan worden in de om wachttijden te verkorten. De proof-of-concept zal bestaan uit
een Real-time Smart Queue Management Systeem voor Wachttijd optimalisatie met Internet of Things (IoT). Tot
slot wordt een evaluatie uitgevoerd waarbij de wachttijden vóór en na de Proof-of-Concept worden vergeleken
om de effectiviteit van de IoT oplossing te beoordelen.
%\lipsum[1-4]
