%%=============================================================================
%% Samenvatting
%%=============================================================================

% TODO: De "abstract" of samenvatting is een kernachtige (~ 1 blz. voor een
% thesis) synthese van het document.
%
% Een goede abstract biedt een kernachtig antwoord op volgende vragen:
%
% 1. Waarover gaat de bachelorproef?
% 2. Waarom heb je er over geschreven?
% 3. Hoe heb je het onderzoek uitgevoerd?
% 4. Wat waren de resultaten? Wat blijkt uit je onderzoek?
% 5. Wat betekenen je resultaten? Wat is de relevantie voor het werkveld?
%
% Daarom bestaat een abstract uit volgende componenten:
%
% - inleiding + kaderen thema
% - probleemstelling
% - (centrale) onderzoeksvraag
% - onderzoeksdoelstelling
% - methodologie
% - resultaten (beperk tot de belangrijkste, relevant voor de onderzoeksvraag)
% - conclusies, aanbevelingen, beperkingen
%
% LET OP! Een samenvatting is GEEN voorwoord!

%%---------- Nederlandse samenvatting -----------------------------------------
%
% TODO: Als je je bachelorproef in het Engels schrijft, moet je eerst een
% Nederlandse samenvatting invoegen. Haal daarvoor onderstaande code uit
% commentaar.
% Wie zijn bachelorproef in het Nederlands schrijft, kan dit negeren, de inhoud
% wordt niet in het document ingevoegd.

\IfLanguageName{english}{%
\selectlanguage{dutch}
\chapter*{Samenvatting}
\lipsum[1-4]
\selectlanguage{english}
}{}

%%---------- Samenvatting -----------------------------------------------------
% De samenvatting in de hoofdtaal van het document

\chapter*{\IfLanguageName{dutch}{Samenvatting}{Abstract}}
De wachttijden bij dienstverlenende organisaties zorgen al jarenlang voor frustratie bij patiënten en cliënten. Ondanks interne optimalisaties en digitalisering blijft het voor velen moeilijk om vlot geholpen te worden, vooral tijdens piekuren. Om die reden richt dit onderzoek zich op de vraag hoe \gls{iot}-technologie kan worden ingezet om de bezettingsgraad van wachtruimtes in real-time te meten en voorspellende analyses toe te passen, met als doel het inzicht in bezettingsgraden te verbeteren en een basis te bieden voor efficiënter bezoekersbeheer. \\

Naast het verbeteren van bezoekersstromen kan het systeem ook input leveren voor triage, personeelsplanning en capaciteitsbeheer, zeker in sectoren waar seizoensinvloeden een rol spelen. \\

Het onderzoek start met een literatuurstudie over bestaande oplossingen. Geïdentificeerde \gls{iot}-apparaten worden geanalyseerd en geselecteerd voor een \gls{poc}, waarbij ook \gls{rtsp}-camera’s voor de nulmeting worden vergeleken. De \gls{poc} bestaat uit drie fasen: een nulmeting met Google Vertex AI Occupancy Analytics, de implementatie van een \gls{iot}-gebaseerd real-time bezettingssysteem en een evaluatie van de resultaten, inclusief voorspellende modellen voor piekmomenten. \\

De resultaten tonen dat het \gls{iot}-systeem technisch haalbaar is en real-time bezettingsdata kan leveren. Vergelijking met Google Vertex AI bevestigt gedeeltelijke betrouwbaarheid, maar storingen en een beperkte dataset beïnvloeden de robuustheid. Voorspellende analyses met ARIMA laten potentieel zien om piekmomenten te anticiperen. \\

Afsluitend toont dit onderzoek dat \gls{iot}-technologie, aangevuld met voorspellende analyses, een veelbelovend en privacyvriendelijk alternatief vormt voor traditionele systemen, hoewel verdere optimalisatie nodig is voor praktische inzetbaarheid.

%\lipsum[1-4]
