%%=============================================================================
%% Samenvatting
%%=============================================================================

% TODO: De "abstract" of samenvatting is een kernachtige (~ 1 blz. voor een
% thesis) synthese van het document.
%
% Een goede abstract biedt een kernachtig antwoord op volgende vragen:
%
% 1. Waarover gaat de bachelorproef?
% 2. Waarom heb je er over geschreven?
% 3. Hoe heb je het onderzoek uitgevoerd?
% 4. Wat waren de resultaten? Wat blijkt uit je onderzoek?
% 5. Wat betekenen je resultaten? Wat is de relevantie voor het werkveld?
%
% Daarom bestaat een abstract uit volgende componenten:
%
% - inleiding + kaderen thema
% - probleemstelling
% - (centrale) onderzoeksvraag
% - onderzoeksdoelstelling
% - methodologie
% - resultaten (beperk tot de belangrijkste, relevant voor de onderzoeksvraag)
% - conclusies, aanbevelingen, beperkingen
%
% LET OP! Een samenvatting is GEEN voorwoord!

%%---------- Nederlandse samenvatting -----------------------------------------
%
% TODO: Als je je bachelorproef in het Engels schrijft, moet je eerst een
% Nederlandse samenvatting invoegen. Haal daarvoor onderstaande code uit
% commentaar.
% Wie zijn bachelorproef in het Nederlands schrijft, kan dit negeren, de inhoud
% wordt niet in het document ingevoegd.

\IfLanguageName{english}{%
\selectlanguage{dutch}
\chapter*{Samenvatting}
\lipsum[1-4]
\selectlanguage{english}
}{}

%%---------- Samenvatting -----------------------------------------------------
% De samenvatting in de hoofdtaal van het document

\chapter*{\IfLanguageName{dutch}{Samenvatting}{Abstract}}

De wachttijden bij dienstverlenende organisaties zorgen al jarenlang voor frustratie bij cliënten. Ondanks interne optimalisaties en digitalisering blijft het voor veel leden moeilijk om vlot geholpen te worden zonder lange wachttijden, vooral tijdens piekuren. Dit leidt tot onnodige frustraties en het gevoel dat men onvoldoende gehoord wordt. Om die reden richt dit onderzoek zich op het beantwoorden van de volgende vraag: Hoe kan IoT-technologie worden ingezet om de bezettingsgraad van wachtruimtes in real-time te meten en voorspellende analyses toe te passen, met als doel het inzicht in wachttijden en de efficiëntie van bezoekersbeheer te verbeteren? \\

Naast het verbeteren van bezoekersstromen en wachttijdbeheer kan het systeem ook waardevolle input leveren voor bredere optimalisatiedoeleinden, zoals triageprocessen, personeelsplanning en capaciteitsbeheer, zeker in sectoren waar seizoensinvloeden een rol spelen. \\

Het onderzoek start met een analyse van de uitgevoerde literatuurstudie over de bestaande oplossingen. Vervolgens worden de in de literatuur geïdentificeerde IoT-apparaten geanalyseerd en wordt hieruit een selectie gemaakt. De devices worden uiteindelijk met elkaar vergeleken om de meest geschikte devices voor de proof-of-concept te selecteren. De proof-of-concept bestaat uit drie fasen. Eerst wordt een nulmeting uitgevoerd met behulp van Google Vertex AI Occupancy Analytics. Vervolgens wordt een IoT-gebaseerd Real-Time Bezettingsmonitoring Systeem geïmplementeerd. Tot slot vindt een evaluatie plaats waarbij de gegevens van zowel de Google Vertex AI Occupancy Analytics als het IoT-gebaseerde Real-Time Bezettingsmonitoring Systeem met elkaar worden vergeleken. \\
%TODO de werkelijke resultaten toevoegen samen met de Aanbevelingen op basis van je resultaten Beperkingen van het onderzoek NADAT DE ONDERZOEK AFGEROND IS

De resultaten zullen aantonen in welke mate IoT-technologie in staat is om nauwkeurige en real-time informatie te leveren over de bezettingsgraad van wachtruimtes. Daarnaast geven ze inzicht in hoe het IoT-gebaseerde systeem zich verhoudt tot het op beeldherkenning gebaseerde systeem van Google Occupancy Analytics.

%\lipsum[1-4]
