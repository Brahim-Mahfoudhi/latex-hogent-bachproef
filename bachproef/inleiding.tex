%%=============================================================================
%% Inleiding
%%=============================================================================

\chapter{\IfLanguageName{dutch}{Inleiding}{Introduction}}%
\label{ch:inleiding}

%De inleiding moet de lezer net genoeg informatie verschaffen om het onderwerp te begrijpen en in te zien waarom de onderzoeksvraag de moeite waard is om te onderzoeken. In de inleiding ga je literatuurverwijzingen beperken, zodat de tekst vlot leesbaar blijft. Je kan de inleiding verder onderverdelen in secties als dit de tekst verduidelijkt. Zaken die aan bod kunnen komen in de inleiding~\autocite{Pollefliet2011}:

\begin{itemize}
  \item context, achtergrond
  \item afbakenen van het onderwerp
  \item verantwoording van het onderwerp, methodologie
  \item probleemstelling
  \item onderzoeksdoelstelling
  \item onderzoeksvraag
  \item \ldots
\end{itemize}

\section{\IfLanguageName{dutch}{Context}{Context}}%
In veel publieke en private dienstverlenende omgevingen zoals ziekenhuizen en overheidskantoren is er onvoldoende inzicht in de bezettingsgraad en wachttijden. Volgens een onderzoek uitgevoerd door \autocite{Jobe2018} liggen de bezettingsgraden in Belgische spoedgevallendiensten tussen 68\% en 100\%, dit wijst op langdurige capaciteitsproblemen tijdens piekuren. Ook \autocite{Lefevre2019} toont aan dat in bepaalde zorgdiensten zoals materniteiten de gemiddelde bezettingsgraad onder de 50\% ligt, met duidelijke verschillen tussen de regio's. Deze cijfers illustreren dat er in de Belgische publieke dienstverlening geen algemene strategie bestaat voor bezettingsbeheer, wat leidt tot een inefficiënt gebruik van de ruimte en moeilijk voorspelbare bezoekersdrukte. Dit toont een duidelijke nood aan real-time monitoring om piekmomenten te detecteren en wachttijden onder controle te houden. De huidige meetmethoden zijn vaak onnauwkeurig, niet real-time, of onvoldoende afgestemd op veranderende bezoekersstromen. Dit kan resulteren in frustraties bij bezoekers en inefficiënt gebruik van middelen. Daarom is er een duidelijke behoefte aan een geautomatiseerd en betrouwbaar meetsysteem dat zowel de huidige situatie als toekomstige drukte in kaart kan brengen. Hoewel IoT-technologie veelbelovend is, blijft het in de praktijk onduidelijk hoe effectief deze oplossingen werkelijk zijn in wachtruimteomgevingen. \DONE


\section{\IfLanguageName{dutch}{Probleemstelling}{Problem Statement}}%
\label{sec:probleemstelling} \DONE
In veel publieke en private dienstverlenende omgevingen, zoals ziekenhuizen, overheidskantoren, zijn wachtruimtes vaak drukbezet zonder inzicht in de bezettingsgraad of wachttijden. Dit maakt het efficiënt beheer van de bezoekersstromen en personeelsinzet moeilijker. \\  

Traditionele meetmethodes zijn vaak handmatig, onnauwkeurig of niet in real-time, dit kan leiden tot frustratie bij bezoekers en inefficiënt gebruik van middelen. Er is dus een duidelijke nood aan geautomatiseerde systemen die real-time inzicht kunnen bieden in deze situatie en toekomstige drukte voorspellen. Hoewel IoT-sensoren beschikbaar zijn, is het niet duidelijk hoe praktisch toepasbaar en betrouwbaar ze zijn in wachtruimtes. Tijdens een gesprek met een medewerker in een dienstverlenende omgeving bleek dat een gebrek aan inzicht in bezettingsgraad en wachttijden een reëel probleem vormt, wat de relevantie van dit onderzoek benadrukt.

\section{Doel van het onderzoek} \DONE
Het doel van dit onderzoek is het bepalen of IoT-technologieën een daadwerkelijke oplossing kan zijn om wachtruimtebezetting in real-time in kaart te brengen, transparant te maken en beter te voorspellen om de bezoekersstromen efficiënter te maken. De uiteindelijke focus ligt op het valideren van de werking van het meetsysteem en het verzamelen van basisdata voor voorspellende analyses, niet op het onmiddellijk verkorten van wachttijden.

\section{\IfLanguageName{dutch}{Onderzoeksvraag}{Research question}}%
\label{sec:onderzoeksvraag} \WORDENDEVRAGENBEANTWOORD?
De centrale onderzoeksvraag in dit onderzoek luidt: "Hoe kan IoT-technologie worden ingezet om de bezettingsgraad van wachtruimtes in real-time te meten en voorspellende analyses toe te passen, met als doel het inzicht in bezettingsgraden te verbeteren en zo een basis te bieden voor efficiënter bezoekersbeheer." Om deze vraag te beantwoorden, is het belangrijk om een antwoord te geven op een aantal deelvragen in het probleem- en oplossingsdomein.  \\ 

\textbf{Probleem domein:}
\begin{itemize}
    \item Welke beperkingen ondervinden dienstverlening instellingen momenteel bij het meten van bezettingsgraden in wachtruimtes?
    \item Waarom is het verkrijgen van accurate en real-time bezettingsdata essentieel voor het optimaliseren van bezoekersstromen?
    \item Welke uitdagingen zijn er bij het handmatig meten en analyseren van wachttijden?
    \item Wat zijn de eisen waaraan een meetsysteem moet voldoen om als betrouwbaar alternatief te dienen?
\end{itemize}

\textbf{Oplossing domein:}
\begin{itemize}
    \item Welke criteria worden gebruikt om sensoren met elkaar te vergelijken in een Proof-of-Concept en hoe worden deze criteria gebruikt om IoT-oplossingen te evalueren ten opzichte van bestaande systemen zoals Google Vertex AI Occupancy Analytics?
    \item Hoe verhoudt de IoT-oplossing zich tot bestaande oplossingen zoals Google Vertex AI Occupancy Analytics?
    %\item Hoe kan Little’s Law toegepast worden op de verzamelde IoT-gegevens om inzicht te krijgen in wachttijden?
    \item Op welke manier draagt voorspellende data-analyse bij aan het optimaliseren van bezoekers- en personeelsbeheer?
    \item Hoe kan de integratie van sensoren met bestaande infrastructuren efficiënt worden gerealiseerd?
\end{itemize}


\section{\IfLanguageName{dutch}{Onderzoeksdoelstelling}{Research objective}}%
\label{sec:onderzoeksdoelstelling}
Dit onderzoek heeft als doel te bepalen hoe IoT-technologie kan worden ingezet om de bezettingsgraad van wachtruimtes real-time te meten en wachttijden te voorspellen. Via de Proof-of-Concept (PoC) wordt nagegaan hoe IoT-sensoren en data-analyse de bezettingsgraad en bezoekersstromen accuraat en privacyvriendelijk in kaart kunnen brengen.


\section{Verwachte eindresultaat} 
Het verwachte eindresultaat is een proof-of-concept die aantoont hoe IoT-oplossingen de bezettingsgraad en wachttijden kunnen meten en inzichtelijk maken. Ook indien het systeem technisch niet volledig operationeel blijkt, levert de PoC waardevolle inzichten op in haalbaarheid, beperkingen en vervolgstappen.


\section{\IfLanguageName{dutch}{Opzet van deze bachelorproef}{Structure of this bachelor thesis}}%
\label{sec:opzet-bachelorproef}

% Het is gebruikelijk aan het einde van de inleiding een overzicht te
% geven van de opbouw van de rest van de tekst. Deze sectie bevat al een aanzet
% die je kan aanvullen/aanpassen in functie van je eigen tekst.

De rest van deze bachelorproef is als volgt opgebouwd:

In Hoofdstuk~\ref{ch:stand-van-zaken} wordt een overzicht gegeven van de stand van zaken binnen het onderzoeksdomein, op basis van een literatuurstudie.

In Hoofdstuk~\ref{ch:methodologie} wordt de methodologie toegelicht en worden de gebruikte onderzoekstechnieken besproken om een antwoord te kunnen formuleren op de onderzoeksvragen.

% TODO: Vul hier aan voor je eigen hoofstukken, één of twee zinnen per hoofdstuk

In Hoofdstuk~\ref{ch:conclusie}, tenslotte, wordt de conclusie gegeven en een antwoord geformuleerd op de onderzoeksvragen. Daarbij wordt ook een aanzet gegeven voor toekomstig onderzoek binnen dit domein.